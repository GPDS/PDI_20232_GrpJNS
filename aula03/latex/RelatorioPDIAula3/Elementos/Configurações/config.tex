% --- 
% CONFIGURAÇÕES
% --- 

% ---
% Reestrutura a fonte do capítulo e da sessão
% ---
\renewcommand{\ABNTEXchapterfont}{\fontseries{b}}
\renewcommand{\ABNTEXchapterfontsize}{\large}
\renewcommand{\ABNTEXsectionfont}{\fontseries{m}}
\renewcommand{\ABNTEXsectionfontsize}{\normalsize}
\renewcommand{\ABNTEXsubsectionfont}{\fontseries{b}}
\renewcommand{\ABNTEXsubsectionfontsize}{\normalsize}
\renewcommand{\ABNTEXsubsubsectionfont}{\fontseries{m}}
\renewcommand{\ABNTEXsubsubsectionfontsize}{\normalsize}
% ---

% ---
% Configurações do pacote backref
% Usado sem a opção hyperpageref de backref
\renewcommand{\backrefpagesname}{Citado na(s) página(s):~}
% Texto padrão antes do número das páginas
\renewcommand{\backref}{}
% Define os textos da citação
\renewcommand*{\backrefalt}[4]{
	\ifcase #1 %
		Nenhuma citação no texto.%
	\or
		Citado na página #2.%
	\else
		Citado #1 vezes nas páginas #2.%
	\fi}%
% ---

% ---
% Informações de dados para CAPA e FOLHA DE ROSTO
% ---
\instituicao {
Instituto Federal da Paraíba\\
Coordenação de Pós-graduação em Engenharia Elétrica\\
Disciplina: Processamento Digital de Imagens
}

\titulo {Relatório\\Atividade Avaliativa da Aula 3}

\tipotrabalho {Relatório}

\autor {Nayana Priscylla Nery dos Santos \\
        Jônatas Mendes da Cruz \\
        Sérgio Ricardo Maciel Pedrosa}
\orientador {Carlos Danilo Miranda Régis, Dr.}
\coorientador{}

\local {João Pessoa - PB}
\date {Setembro de 2023}

\preambulo{\textbf{Relatório apresentado ao professor Carlos Danilo Miranda Régis, referente a atividade avaliativa da terceira aula da disciplina de Processamento Digital de Imagens, do Programa de Pós-Graduação em Engenharia Elétrica do Instituto Federal de Educação, Ciência e Tecnologia da Paraíba}}
% ---

% ---
% Folha da capa
% ---
\renewcommand{\imprimircapa}
{
  \begin{capa}
    \begin{center}
        \includegraphics[width=15.5cm,keepaspectratio]{Elementos/Figuras/Figuras capa/header.jpg}
        \large\imprimirinstituicao
        \vspace{5cm}

       	\large\imprimirautor
    
        \vspace*{\fill}
       		\large\imprimirtitulo
        \vspace*{\fill}
        
        \normalsize\imprimirlocal
        \par
        \normalsize\imprimirdata
    \end{center}
  \end{capa}
}
% ---

% ---
% Folha de rosto
% ---
\renewcommand{\imprimirfolhaderosto}
{
    \begin{folhaderosto}
        \begin{center}
            Instituto Federal da Paraíba\\
            Programa de Pós-graduação em Engenharia Elétrica\\
            Disciplina: Ferramentas para Análise e Processamento de Sinais\\

            \imprimirautor

            \vspace*{\fill}
            {\large\imprimirtitulo}
            \vspace*{\fill}
        
            \hspace{.45\textwidth}
            \begin{minipage}
                {.5\textwidth}
                \imprimirpreambulo
            \end{minipage}
               
            \vspace*{\fill}
       	    \textbf{Orientador:} \large\imprimirorientador
            \vspace*{\fill}
        
            \normalsize\imprimirlocal
            \par
            \normalsize\imprimirdata
        \end{center}
    \end{folhaderosto}
}
% ---

% ---
% Altera as margens padrões
% ---
\setlrmarginsandblock{3cm}{2cm}{*}
\setulmarginsandblock{3cm}{2cm}{*}
\checkandfixthelayout
% ---

% --- 
% Espaçamentos entre linhas e parágrafos 
% --- 

% O tamanho do parágrafo é dado por:
\setlength{\parindent}{1.3cm}

% Controle do espaçamento entre um parágrafo e outro:
\setlength{\parskip}{0.2cm}  % tente também \onelineskip

% ---
% compila o índice
% ---
\makeindex
% ---